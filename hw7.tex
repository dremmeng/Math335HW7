% LaTeX Article Template - customizing page format
%
% LaTeX document uses 10-point fonts by default.  To use
% 11-point or 12-point fonts, use \documentclass[11pt]{article}
% or \documentclass[12pt]{article}.
\documentclass{article}

% Set left margin - The default is 1 inch, so the following 
% command sets a 1.25-inch left margin.
\setlength{\oddsidemargin}{0.25in}

% Set width of the text - What is left will be the right margin.
% In this case, right margin is 8.5in - 1.25in - 6in = 1.25in.
\setlength{\textwidth}{6in}

% Set top margin - The default is 1 inch, so the following 
% command sets a 0.75-inch top margin.
\setlength{\topmargin}{-0.25in}

% Set height of the text - What is left will be the bottom margin.
% In this case, bottom margin is 11in - 0.75in - 9.5in = 0.75in
\setlength{\textheight}{8in}
\usepackage{fancyhdr}
\usepackage{float}
\usepackage{mathtools}
\usepackage{amsmath}
\usepackage{amssymb}
\usepackage{graphicx}
\graphicspath{ {./} }

\setlength{\parskip}{5pt} 
\pagestyle{fancyplain}
% Set the beginning of a LaTeX document
\begin{document}

\lhead{Drew Remmenga MATH 335}
\rhead{HW \#6}
%\lhead{Independent Study}
%\rhead{R Lab}

\begin{enumerate}
\item
	\begin{enumerate}
	\item
		\begin{equation*}
		\begin{split}
		Z & = \frac{6-2}{2\sqrt{2}}  \\
		Z & = \sqrt{2} \\
		P(x > Z) & = .07927 \\
		\end{split}
		\end{equation*}
	\item
		Normal with:
		\begin{equation*}
		\begin{split}
		\mu & = \frac{8X+4*2}{4+8} \\
		\mu & = \frac{2X+2}{3} \\
		\sigma^{2} & = \frac{4*8}{8+4} \\
		\sigma^{2} & = \frac{32}{12} \\
		\sigma^{2} & = \frac{8}{3} \\
		\end{split}
		\end{equation*}
	\item
		\begin{equation*}
		\begin{split}
		\mu & = \frac{2(2)+2}{3} \\
		\mu & = 2 \\
		Z & = \frac{6 - 2}{\sqrt{\frac{8}{3}}} \\
		Z & = 2.449 \\
		P(x > 2.449) & = .00714 \\
		\end{split}
		\end{equation*}
	\item
		\begin{equation*}
		\begin{split}
		\hat{\theta} & = \frac{2X+2}{3} \\
		\end{split}
		\end{equation*}
	\item
		Yes. Gaussian likelihood with Gaussian Prior and Gaussian Posterior.
	\end{enumerate}
\item
	\begin{enumerate}
	\item
		\begin{equation*}
		\begin{split}
		\mu & = \frac{1}{1+3} \\
		\mu & = \frac{1}{4} 
		\end{split}
		\end{equation*}
	\item
		\begin{equation*}
		\begin{split}
		\alpha & = 1+X \\
		\beta & = 3+20 - X \\
		\beta & = 23 - X \\
		\end{split}
		\end{equation*}
	\item
		\begin{equation*}
		\begin{split}
		\hat{\theta} & = \frac{\bar{X}}{20} \\
		\hat{\theta} & = \frac{X}{20} \\
		\end{split}
		\end{equation*}
	\item
		\begin{equation*}
		\begin{split}
		\hat{\theta}_{Bays'} & = \frac{1+X}{24} \\
		\hat{\theta}_{Bays'}  & = \hat{\theta}_{MLE} \\
		\frac{X}{20} & = \frac{1+X}{24} \\
		24X & = 20 + 20X \\
		X & = 5 \\
		\hat{\theta}_{Bays'} & < \hat{\theta}_{MLE}, X \in  (5, \infty)\\
		\hat{\theta}_{Bays'} & > \hat{\theta}_{MLE}, X \in (-\infty, 5)
		\end{split}
		\end{equation*}
	\item
		\begin{equation*}
		\begin{split}
		\hat{\theta}_{Bays'} & = \frac{1+X}{24} \\
		\end{split}
		\end{equation*}
	\end{enumerate}
\item
		\begin{equation*}
		\begin{split}
		\Gamma(1) & = 1 \\
		\Gamma(2) & = 1 \\
		B(1,1) & = \frac{x^{1-1}(1-x)^{1-1}}{\frac{\Gamma (1) \Gamma (1)}{\Gamma (1+1)}} I(x)_{0,1} \\
		B(1,1) & = \frac{(1)(1)}{\frac{(1)(1)}{(1)}}I(x)_{0,1} \\
		B(1,1) & = 1 I(x)_{0,1} \\
		B(1,1) & = U(0,1) 
		\end{split}
		\end{equation*}
\item
	\begin{enumerate}
	\item

 X ~ Bin(50,$\theta$), X = 3, U(0,1) = B(1,1) 

		\begin{equation*}
		\begin{split}
		B(\alpha + X, \beta + n - x) & = B(1+3, 1+50-3) \\
		& = B(4, 48) \\ 
		\end{split}
		\end{equation*}
	
	\item
		\begin{equation*}
		\begin{split}
		\hat{\theta}_{Bays'} & = \frac{4}{52} \\
		\end{split}
		\end{equation*}
	\end{enumerate}
\end{enumerate}

\end{document}
